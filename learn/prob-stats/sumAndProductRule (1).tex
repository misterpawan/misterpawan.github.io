
\begin{frame}{Rule of Sum using Set Language}
\pause
\begin{alertblock}{Rule of Sum}
If there is a set $A$ with $k$ elements, a set $B$ with $n$
elements and these sets do not have common
elements, then the set $A \cup B$ has $n+k$ elements
\end{alertblock} \pause 
\begin{figure}
\includegraphics[scale=0.35]{sum}
\end{figure}
\end{frame}


\begin{frame}{Remark on Rule of Sum}
\pause 
\begin{alertblock}{Rule of sum}
Can we apply rule of sum when $A$ and $B$ intersect as follows? \pause 
\begin{figure}
\includegraphics[scale=0.28]{sum2}
\end{figure} \pause 
\end{alertblock}
\begin{itemize}
\item If we consider $|A|+|B|$ as in sum rule, then we will be wrong \pause 
\item We will count elements that belong to both $A$ and $B$ twice
\item $|A \cup B| = |A| + |B| - |A \cap B|$ (Inclusion-Exclusion Principle)
\end{itemize}
\end{frame}


\begin{frame}{Applications of sum rule}
\pause 
\begin{alertblock}{Sum rule}
Count all integers from 1 to 10 that are divisible by 2 or by 3
\end{alertblock} \pause 
\begin{itemize}
\item Let us count all the numbers from 1 to 10:
\begin{align*}
1,2,3,4,5,\alert{6},7,8,9,10
\end{align*} \pause 
\item Here 6 is divisible both by 2 and by 3. Hence, rule of sum can't be applied!
\end{itemize}
\end{frame}


\begin{frame}{Sum Rule: Example}
\pause 
\begin{alertblock}{Number of Paths}
Suppose there are several points connected by
arrows. There is a starting point $s$ (called source)
and a final point $t$ (called sink). How many
different ways are there to get from $s$ to $t?$
\end{alertblock} \pause 
\begin{figure}
\includegraphics[scale=0.35]{count}
\end{figure} \pause 
\begin{itemize}
\item counting can be done recursively \pause 
\item for each node count the number of paths from $s$ to this node
\begin{itemize}
\item sum rule will be used
\end{itemize}
\end{itemize}
\end{frame}

\section{Product Rule}

\begin{frame}{Product Rule}
\pause 
\begin{alertblock}{Product Rule}
If there are $k$ object of the first type and there are
$n$ object of the second type, then there are $k \times n$
pairs of objects, the first of the first type and the
second of the second type
\end{alertblock} \pause 
\begin{figure}
\includegraphics[scale=0.35]{mult}
\end{figure} \pause 
\begin{itemize}
\item Hence, there are $4 \times 3 = 12$ options
\end{itemize}
\end{frame}


\begin{frame}{All possible pairs}
\pause 
\begin{figure}
\includegraphics[scale=0.38]{mult2}
\end{figure}
\end{frame}


\begin{frame}{Rule of Product Using Sets}
\pause 
\begin{alertblock}{Product Rule}
If there is a finite set $A$ and a finite set $B,$ then
there are $|A| \times |B|$ pairs of objects, the first from
$A,$ and the second from $B$
\end{alertblock}
\end{frame}

\begin{frame}{Tuples: Application of Product Rule}
\pause 
\begin{alertblock}{Tuples}
How many different 5-symbol passwords can we
create using lower case Latin letters only? (the
size of the alphabet is 26)
\end{alertblock}
\pause 
\begin{itemize}
\item How many different 1-letter passwords are possible? \pause 
\item What about 2-letters? \pause 
\item How many different 3-letter words are possible? \pause 
\item Can you now answer the question above?
\end{itemize}
\end{frame}
