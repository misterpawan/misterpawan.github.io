\section[Counting]{Basic Counting}

%\section{Basic Counting}

%\begin{comment}

\begin{frame}{Why Counting?}
\pause 
\begin{itemize}
\item counting is a basic task in mathematics \pause 
\item \yellow{Our objective:} To tell how many objects are there without actually counting \pause 
\item \yellow{applications of counting:} \pause 
\begin{itemize}
\item number of steps of an algorithm \pause 
\item estimating probability of occurrence of an event (\yellow{in this course}) \pause 
\item proofs such as Pigeon Hole Principle (PHP) 
\end{itemize}
\end{itemize}
\end{frame}

\begin{frame}{Brief History of Counting}
\pause 
\begin{columns}
\begin{column}{0.5\textwidth}
\begin{figure}
\includegraphics[scale=0.33]{bone4}
\end{figure}
\end{column}
\pause 
\begin{column}{0.5\textwidth}
\begin{itemize}
\item \alert{Ishango Bone} is possibly the \yellow{oldest} mathematical artefact still in existence \pause 
\item It is dates back to the Upper Paleolithic period of human history, and is approximately \yellow{20,000 years old.} \pause 
\item The bone is 10 cm long and contains a series of notches, which many scientists believe were used for counting.
\end{itemize}
\end{column}

\end{columns}
\end{frame}


\begin{frame}{Brief History of Counting}
\begin{columns}
\begin{column}{0.4\textwidth}
\begin{figure}
\includegraphics[scale=0.33]{mult200}
\end{figure}
\end{column}
\pause 
\begin{column}{0.65\textwidth}
\begin{itemize}
\item This tablet shows a \yellow{multiplication table} that was created around \yellow{2600 BCE} in the Sumerian city of Shuruppak \pause 
\item The table has three columns. The dots in the first two columns represent distances ranging from around 6 meters to 3 kilometres. The third column contains the product of the first two \pause 
\item Sumer was a region of ancient Mesopotamia in the Middle East
\end{itemize}
\end{column}

\end{columns}
\end{frame}


\begin{frame}{How do we count?}
%    \begin{alertblock}{Question-1}
%    How many numbers are there between 1 and 10?
%    \end{alertblock}
%    \pause 
%    \begin{alertblock}{Answer}
%     10
%    \end{alertblock}
    \pause 
    \begin{alertblock}{Question-2}
    How many numbers are there between 21 and 31?
    \end{alertblock}
    \pause 
    \begin{alertblock}{Answer}
     11
    \end{alertblock}
    \pause 
    \begin{alertblock}{Question-3}
    How many numbers are there between $m$ and $n, \: m<n?$
    \end{alertblock}
    \pause 
    \begin{alertblock}{Answer}
     $m-n+1$
    \end{alertblock}
    
\end{frame}

\begin{frame}{How do we count?}
    \begin{alertblock}{Question-1}
    How many numbers between 33 and 67 are divisible by 4?
    \end{alertblock}
    %\pause 
    \vspace{5cm}
%    \begin{alertblock}{Answer}
%     
%    \end{alertblock}
    
\end{frame}

\section{Sum Rule}

\begin{frame}{Sum Rule: Statement and Example}
\pause 
\begin{alertblock}{Sum Rule}
If there are $m$ objects of first type and there are $n$ objects of second type, then 
there are $n+m$ objects of one of the two types.
\end{alertblock}

\pause 

\begin{alertblock}{Sum Rule Example}
How many of the Pizza or Burgers places are there?
\begin{figure}
\includegraphics[scale=0.25]{sum1.png}
\end{figure}
\end{alertblock}
\pause 
\begin{alertblock}{Answer}
There are 7 Pizzas and there are 5 Burgers, hence, by sum rule, we have
7 + 5 = 12
\end{alertblock}

\end{frame}


\begin{frame}{Another Sum Rule Example}
\pause 
\begin{columns}
\begin{column}{0.6\textwidth}
\begin{alertblock}{Example}
A piece stays in the bottom left corner of a
chessboard. In one move it can move one step to
the right or one step up. How many moves are
needed to get to the position on the picture?
\end{alertblock}
\end{column}
\pause 
\begin{column}{0.4\textwidth}
\begin{figure}
\includegraphics[scale=0.3]{chess}
\end{figure}
\end{column}
\end{columns}
\pause 
\begin{itemize}
\item There are \yellow{two} types of moves: \pause move right and move up \pause 
\item To get to the column 4, we need 3 moves to the right \pause 
\item To get to the row 6, we need 5 moves up \pause 
\item \yellow{Applying sum rule:} In total, we need 3+5=8 moves
\end{itemize}
\end{frame}

\begin{frame}{Caution using Sum Rule!}
\pause 
\begin{alertblock}{Question}
Count all integers from 1 to 10 that are divisible by 2 or by 3
\end{alertblock}

\pause

\begin{alertblock}{Answer}
\begin{itemize}
\item There are 5 numbers divisible by 2: \quad 2, 4, 6, 8, 10 \pause 
\item There are 3 numbers divisible by 3: \quad 3, 6, 9 \pause 
\item Hence, by sum rule, the answer is: \quad 5+3 = 8
\end{itemize} \pause 
Is this correct answer? \pause \yellow{No} \\ \pause 
Let us count directly: 2,3,4,6,8,9,10, that is the answer is 7
\end{alertblock}
\pause 
\begin{alertblock}{Caution with Sum Rule}
In the rule of sum, \yellow{no object should belong to both types!}
\end{alertblock}

\end{frame}












