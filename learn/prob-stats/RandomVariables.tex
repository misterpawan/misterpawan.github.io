\section{Random Variables}
%%
%%
%%
%
%
%
%
%
%
%
%
%
%
%
%
%
%
%
\begin{frame}{Random Variables}
\pause 
\begin{alertblock}{Examples of typed variables in C}
In some languages, such as, C/C++. we have the concept of a \yellow{typed variable}: \pause 
\begin{itemize}
\item \yellow{int} i = 4;
\item \yellow{float} x = 10;
\item \yellow{char} y = 'x';
\end{itemize}
\end{alertblock}
\pause 
\begin{alertblock}{Examples of random variable}
Let $X$ denote the outputs after we roll a die, then 
\[ X = 3 \]
means that after rolling a die, we obtain 3 as output. 

\pause Since the number that is going to be assigned to \yellow{variable} $X$ is going to be \yellow{random}, it is called \yellow{random variable}.
\end{alertblock}

\end{frame}

%
%
%
%
%
%
%
%
%
%
%
%
%
%
%
%
%
%
%
%
%

\begin{frame}{Define Random Variable}
\pause 
\begin{alertblock}{Definition of Random Variable}
A \yellow{random variable} $X$ is a function from the sample space to the real numbers. 
\[ X:S \rightarrow \mathbb{R}. \] \pause 
\begin{itemize}
\item We usually denote random variables by \yellow{capital} letters: $X,Y, \dots$ \pause 
\item Random variable is a function with \yellow{domain} $S$ and \yellow{co-domain} $\mathbb{R}$ \pause 
\item The range of a random variable is the set of possible values of $X$
\end{itemize}
\end{alertblock}
\pause 
\begin{alertblock}{Examples of Random Variables...}
Find the range of the following random variables:
\begin{itemize}
\item I toss a coin 10 times. Let X be the number of heads I observe \pause 
\item I toss a coin until the first tail appears. Let $Y$ be the total number of coin tosses
\end{itemize}
\end{alertblock}
\end{frame}

%
%
%
%
%
%
%
%
%
%
%
%
%
%
%
%
%
%
%


\begin{frame}{Quiz on Random Variable...}
\pause 
\begin{alertblock}{Quiz on Random Variable}
Consider an Experiment: 3 coins are flipped. Let $X$ be the number of tails. 
Answer the following: \pause 
\begin{itemize}
\item What is the value of $X$ for the following outcomes?  \pause 
\begin{itemize}
\item $(H,H,H)$
\item $(T,T,H)$
\end{itemize} \pause 
\item What is the event when $X=2?$ \pause 
\item What is $P(X=2)?$
\end{itemize}
\end{alertblock}
\end{frame}


%
%
%
%
%
%
%
%
%
%


\begin{frame}{Random Variables are Not Events!}
\pause
\begin{alertblock}{Remarks on Random variables}
\begin{itemize}
\item random variables are \yellow{not} events! \pause 
\item when a random variable is \yellow{assigned} a value, then it becomes event
\end{itemize}
\end{alertblock}
\vspace{1cm}

\begin{table}
\begin{tabular}{lll}
\hline 
$X=x$ & Set of Outcomes & $P(X=k)$ \\   \hline 
$X=0$ & $\{ (T,T,T) \}$ & $1/8$ \\  
$X=1$ & $\{ (H,T,T), (T,H,T), (T,T,H) \}$ & $3/8$ \\  
$X=2$ & $\{ (H,H,T), (H,T,H), (T,H,T) \}$ & $3/8$ \\ 
$X=3$ & $\{ (H,H,H) \}$ & $1/8$ \\  
$X \geq 4$ & $\{ \:	 \}$ & $0$ \\  
\end{tabular}
\caption{Consider an experiment where 3 coins are flipped, and $X$ denotes number of heads}
\end{table}

\end{frame}



%
%
%
%
%
%


\begin{frame}{Discrete Random Variables...}
\pause

\begin{alertblock}{Recall: countable sets}
A set $A$ is \yellow{countable} if either it is a \yellow{ finite} set, or it can be put in \yellow{1-1 correspondence} with set of natural numbers. 
\end{alertblock}
\pause 

\begin{alertblock}{Discrete Random Variables}
A random variable $X$ is called \yellow{discrete random variable}, if its range is \yellow{countable}. 
\end{alertblock}
\pause 

\begin{alertblock}{Types of Random Variables...} \pause 
There are three types of random variables: \pause 
\begin{enumerate}
\item discrete random variables \pause 
\item continuous random variables \pause 
\item mixed random variables
\end{enumerate}
\end{alertblock}


\end{frame}


%
%
%
%
%
%
%
%
%


\begin{frame}{Examples of Types of Random Variables...}
\pause

\begin{alertblock}{Examples of random variables}
\begin{enumerate}
\item I toss a coin 100 times. Let $X$ be the number of heads I observe \pause 
\item I toss a coin until the first heads appears. Let $Y$ be the total number of coin tosses \pause 
\item Random variable $T$ defined as the time (in hours) from now until next earthquake occurs in a certain city \pause 
\item Let $X$ be the height of students in a class
\end{enumerate}
\end{alertblock}

\end{frame}




%
%
%
%
%
%
%



\begin{frame}{Probability Mass Function...}
\pause 
\begin{alertblock}{Definition of probability mass function and some remarks...}
Let $X$ be a random variable with range $R_X = \{ x_1, x_2, \dots, \},$ which is \yellow{finite or countably infinite}. \pause The function 
\[ P_X(x_k) = P(X = x_k), \: \text{for}~k=1,2,3, \dots \] 
is called \yellow{probability mass function (PMF)} of $X.$
\pause
\begin{itemize}
\item Hence, probability mass function $P_X(x_k)$ is a \yellow{shorthand} for $P(X = x_k)$ \pause
\item The subscript in $P_X(x_k)$ indicates that it is the PMF of random variable $X$ \pause 
\item For \yellow{discrete} random variable, PMF is also called \yellow{probability distribution} \pause 
\item The term probability distribution function is almost always reserved for \yellow{cumulative } distribution(to be introduced)	
\end{itemize}

\end{alertblock}

\end{frame}



\begin{frame}{Example of PMF...}
\pause
\begin{alertblock}{Examples of PMF}
We toss a \yellow{fair} coin twice. \pause Let $X$ be the random variable denoting the number of heads we observe. \pause 
\begin{enumerate}
\item Find the range of random variable $X,$ i.e., $R_X$ \pause 
\item Find the PMF of random variable $X$
\end{enumerate}
\end{alertblock}
\pause 
\begin{block}{Answer}
\begin{itemize}
\item Sample space $S = \{ HH, HT, TH, TT \}.$ No. of heads: 0,1,2. Hence, $R_X = \{ 0,1,2 \}$ \pause 
\item Since the range is finite set, thus countable, $X$ is a discrete random variable \pause 
\item We now find the PMF of $X: P_X (k) = P(X = k)~\text{for}~k=0,1,2 $ \pause 
\begin{align*}
P_X (0) &= P(X = 0) = P(TT) = 1/4 \\
P_X(1) &= P(X=1) = P(\{ HT, TH \}) = 1/4 + 1/4 = 1/2 \\
P_X(2) &= P(X = 2) = P(HH) = 1/4
\end{align*}
\end{itemize}
\end{block}

\end{frame}

%
%
%
%
%
%
%
%
%


\begin{frame}
\begin{figure}
\includegraphics[scale=0.5]{pmf1}
\end{figure} \pause 
\begin{itemize}
\item Consider an experiment of rolling a six-sided die \pause 
\item The sample space or \yellow{support} of $X$ is $\{ 1,2,3,4,5,6 \}$ \pause 
\item Here $X$ is a \yellow{discrete} random variable, and PMF of $X$ is \pause 
\begin{align*}
P_X(x) = \begin{cases}
&1/6, \quad x \in \{ 1,2,3,4,5,6\} \\
&0, \quad \text{otherwise}
\end{cases}
\end{align*}
\end{itemize}
\end{frame}



\begin{frame}{Example of PMF...}
\pause 
\begin{alertblock}{Problem on PMF}
Consider an \yellow{unfair} coin for which $P(H)=p.$ \pause We toss the coin repeatedly until we observe a head for the first time. \pause Let $Y$ be the total number of times the coin was tossed. \pause 
\begin{enumerate}
\item Is $Y$ a discrete random variable? \pause 
\item Find PMF of the random variable $Y$
\end{enumerate}

\end{alertblock}

\begin{block}{Answer to the problem}
We have 
\begin{align*}
P_Y(1) &= P(Y=1) = P(H) = p \\
P_Y(2) &= P(Y=2) = P(TH) = (1-p)p \\
&\vdots \\
P_Y(k) &= P(Y=k) = P(TT\cdots TH) = (1-p)^{k-1} p
\end{align*}
\end{block}

\end{frame}





\begin{frame}{Properties of PMF...}
\pause 

\begin{alertblock}{Properties of PMF}
\begin{enumerate}
\item $0 \leq P_X(x) \leq 1$ for all $x$ \pause 
\item $\sum_{x \in R_X} P_X(x) = 1$  \pause 
\item For any set $A \subset R_X, P(X \in A) = \sum_{x \in A} P_X(x)$
\end{enumerate}
\end{alertblock}
\vspace{5cm}

\end{frame}


%
%

\begin{frame}{Check Properties of PMF...}
\pause 
\begin{alertblock}{Problem on PMF}
Consider an \yellow{unfair} coin for which $P(H)=p.$ \pause We toss the coin \yellow{repeatedly} until we observe a head for the first time. \pause Let $Y$ be the total number of times the coin was tossed. \pause 
\begin{enumerate}
\item Check $\sum_{y \in R_Y} P_Y(y)=1,$ \quad here $R_Y$ is the \yellow{range} of random variable $Y$ \pause 
\item If $p=1/2, $ \quad find $P(2 \leq Y < 5)$
\end{enumerate}

\end{alertblock}
\vspace{4cm}


\end{frame}


\begin{frame}{Independent Random Variables...}
\pause 

\begin{alertblock}{Independent Random Variables}
We consider two random variables \yellow{independent} if \pause 
\[ P(X=x, Y=y) = P(X=x)P(Y=y), \quad \text{for all}~x,y \] \pause 
In general, if two random variables are \yellow{independent}, then \pause 
\[ P(X \in A, X \in B) = P(X \in A) P(Y \in B), \quad \text{for all sets}~A~\text{and}~B \]
Also, we have \pause 
\[ P(Y=y \mid X = x) = P(Y=y), \quad \text{for all}~x,y \]
\end{alertblock}

\end{frame}


%
%
%

\begin{frame}{Example: independent random variables, PMF}
\pause 
\begin{alertblock}{Problem}
We toss a coin twice, and let $X$ denote the number of heads we observe. \pause After this, we then toss the coin two more times and define $Y$ to be the number of heads we observe. \pause What is 
$P((X<2) ~\text{and}~ (Y>1))?$
\end{alertblock}
\vspace{5cm}

\end{frame}


\section{Expectation}
%%
%%
%%
%%
%%
%%
%%
%%
%%
%%
%%
%%
%%

\begin{frame}{Definition of Expectation..}
\pause 

\begin{alertblock}{Definition of Expectation}
The \yellow{expectation} of a \yellow{discrete} random variable $X$ is defined as \pause 
\begin{align*}
E[X] = \sum_{x:p(x)>0} p(x) \cdot x
\end{align*} \pause 
\begin{itemize}
\item Note we sum over all values of $X$ that have \yellow{non-zero} probability \pause 
\item Other names of expectations: mean, expected value, weighted average, center of mass, first moment
\end{itemize}
\end{alertblock}

\end{frame}



%%
%%
%%
%%
%%
%%
%%


\begin{frame}{Example of Expectation...}
\pause 

\begin{alertblock}{Example of Expectation...}
What is the \yellow{expected} value of 6-sided die roll?
\end{alertblock}
\pause 

\begin{block}{Answer to the question}
\begin{itemize}
\item \yellow{Define a random variable:} $X = \text{RV for value of a roll}$ \pause 
\item \yellow{Find PMF:}  \pause 
\begin{align*}
P(X=x) = \begin{cases}
      1/6, \quad &x \in \{ 1, \dots, 6 \} \\
      0, \quad &\text{otherwise}
\end{cases}
\end{align*} \pause 
\item \yellow{Calculate Expectation:} \pause 
\begin{align*}
E[X] = 1 \left( \dfrac{1}{6}  \right) + 2 \left( \dfrac{1}{6}  \right) + 3 \left( \dfrac{1}{6}  \right) + 4 \left( \dfrac{1}{6}  \right) + 5 \left( \dfrac{1}{6}  \right) + 6 \left( \dfrac{1}{6}  \right) = \left( \dfrac{7}{2} \right) = 3.5
\end{align*}
\end{itemize}
\end{block}

\end{frame}



\begin{frame}{Recall: Linear Functions...}
\pause 

\begin{alertblock}{Definition of linear function}
Let $A,B$ be two \yellow{vector spaces.} \pause A function $f:A \rightarrow B$ is called \yellow{linear} function if the following holds: \pause 
\begin{itemize}
\item $f(u+v) = f(u)+f(v)$ \pause 
\item $f(cu) = c f(u), c$ is \yellow{scalar}
\end{itemize}
\pause 
Equivalent definition is: \pause 
\begin{align*}
f(\alpha u + \beta u) = \alpha f(u) + \beta f(v), \quad \forall u,v \in A, \quad \alpha,\beta~\text{scalars}
\end{align*}
\end{alertblock}

\end{frame}


%%
%%
%%
%%
%%
%%
%%
%%
%%
%%
%%
%%
%%
%%
%%
%%
%%
%%


\begin{frame}{Quiz on linear functions...}
\pause 
\begin{alertblock}{Problem on linear function}
Which of the following functions are linear? \pause 
\begin{enumerate}
\item $f(x) = x$ \pause 
\item $f(x) = ax + b$  \pause 
\item $f(x) = 10$ \pause 
\item $f(x) = Ax + b$ \pause 
\item $f(x) = 3\sin x$ \pause 
\item $f(x,y) = 3x + 4y$ \pause 
\item $f(x,y,z) = 3 + x + y + z$ \pause 
\item $f \begin{pmatrix}
x \\ y \\ z
\end{pmatrix} = \begin{pmatrix}
x + y \\
y - z \\ 
z
\end{pmatrix}$
\end{enumerate}
\end{alertblock}

\end{frame}

%%
%%
%%
%%
%%
%%
%%
%%
%%
%%
%%
%%
%%
%%
%%
%%
%%
%%


\begin{frame}{Properties of Expectation...}
\pause 


\begin{columns}
\begin{column}{0.5\textwidth}
\begin{alertblock}{Linearity of Expectation}
\[ E[aX + b] = a E[X] + b \]
\end{alertblock}
\pause 
\end{column}
\begin{column}{0.5\textwidth}
Let $X =$ 6 sided dice roll. Let $Y = 2X-1.$ If $E[X]=3.5,$ then what is 
$E[Y]=?$
\end{column}
\end{columns}

\pause 


\begin{columns}
\begin{column}{0.5\textwidth}
\begin{alertblock}{Expectation of a sum}
\[ E[X+Y] = E[X] + E[Y] \]
\end{alertblock}
\pause 
\pause 
\end{column}
\begin{column}{0.5\textwidth}
Consider two dice rolls. Let $X$ = roll of dice 1, $Y=$ roll of die 2, then 
$E[X+Y] = ?$
\end{column}
\end{columns}

\pause 



\begin{columns}
\begin{column}{0.5\textwidth}
\begin{alertblock}{Expectation of a function of random variable}
\[ E[g(X)] = \sum_x g(x) p(x) \]
\end{alertblock}
\pause 
\end{column}
\begin{column}{0.5\textwidth}
Let $X =$ 6 sided dice roll. Let $g(x) = x^2.$ If $E[X]=3.5,$ then what is 
$E[g(X)]=?$
\end{column}
\end{columns}

\end{frame}



%%
%%
%%
%%
%%
%%
%%
%%
%%
%%
%%
%%



\begin{frame}{Proof of Linearity of Expectation...}
%\pause 

\begin{center}
\fbox{ $E[aX + b] = a \, E[X] + b$ }
\end{center}

\vspace{6cm}

\end{frame}


%%
%%
%%
%%
%%
%%
%%
%%
%%
%%



\begin{frame}{Proof of Expectation of Function of a Random Variable...}
%\pause 
\begin{center}
\fbox{ $E[g(X)] = \sum_x g(x) \, p(x)$ }
\end{center}

\vspace{6cm}

\end{frame}

%%
%%
%%
%%
%%
%%
%%
%%


\begin{frame}{Quiz on Expectation...}
\pause 

\begin{alertblock}{Question on Expectation}
Let $X$ be a discrete random variable, whose PMF is given as follows
\begin{align*}
P_X(x_k) = \begin{cases}
    \dfrac{1}{3}, \quad \text{for}~x \in \{-1, 0, 1 \}, \\
    0, \quad \text{otherwise}. 
\end{cases}
\end{align*}
Let $Y = |X|.$ What is $E[Y]?$
%\begin{enumerate}
%\item 0
%\item $\dfrac{4}{5}$
%\item $\dfrac{2}{3}$
%\item $\dfrac{1}{3}$
%\end{enumerate}
\end{alertblock}

\end{frame}

\begin{frame}{Expectation and Game of Chance...}
\pause 

\begin{alertblock}{Expectation and Game of Chance Problem}
Consider a game where a die is rolled. After the roll, the total money you gain 
is the output of the die. How much money would you like to pay to play this game? 
\end{alertblock}

\vspace{5cm}

\end{frame}

\begin{frame}{Expectation versus Reality}
    \begin{figure}
        \centering
        \includegraphics[scale=0.4]{expectation}
        % \caption{Caption}
        % \label{fig:my_label}
    \end{figure}
\end{frame}


%%

\section{Saint Petersberg Paradox}

\begin{frame}{Saint Petersberg Paradox...}
\pause 

\begin{alertblock}{Saint Petersberg Paradox}
Consider a fair coin. \pause We toss the coin, if it is heads,the game ends, and you win Rs 2. \pause If it turns out to be tails, I toss it again. \pause If it is heads on the second toss, you win Rs 4. \pause If not, I toss it again, and if it is heads on the third toss,you win Rs 8. \pause In general, if the first heads appears at the $n$th toss, then you win Rs $2^n.$ \pause

There is an entrance fee to play this game. What is the maximum amount of money you would pay to play this game?
\end{alertblock}

\end{frame}


%%
%%

\begin{frame}{Analyzing Solution to Saint Petersberg Paradox Problem...}
\pause 
\begin{columns}
\begin{column}{0.4\textwidth}
\vspace{-0.8cm}
\begin{table}
\begin{tabular}{llll}
$n$ & Prize & Prob & Utility \\ \hline 
1 & Rs 2 & 1/2 & Rs 2 \\
2 & Rs 4 & 1/4 & Rs 4 \\
3 & Rs 8 & 1/8 & Rs 8 \\
4 & Rs 16 & 1/16 & Rs 16 \\
5 & Rs 32 & 1/32 & Rs 32 \\
6 & Rs 64 & 1/64 & Rs 64 \\
7 & Rs 128 & 1/128 & Rs 128 \\
8 & Rs 252 & 1/252 & Rs 252 \\
\vdots & \vdots & \vdots & \vdots\\
\end{tabular}
\end{table}
\end{column} \pause 

\begin{column}{0.6\textwidth}
Let us make some observations. \pause 
\begin{itemize}
\item If we pay Rs 1, then if the game ends in 1st toss, then we get Rs 2. That is we gain Rs 1. \pause  
\item If we pay say Rs 10, and the game ends in 3rd toss, then we get Rs 8, and we loose Rs 2. 
\item Naturally, we wonder what is the expected value? 
\end{itemize}
\end{column}
\end{columns}
\begin{alertblock}{Expected Value for Saint Petersberg Paradox...}
\vspace{-0.5cm}
\begin{align*}
E[X] &= 1 \times \dfrac{1}{2} + 4 \times \dfrac{1}{4} + 8 \times \dfrac{1}{8} \dots = 1 + 1 + 1 + \dots = \infty
\end{align*} \pause 
So, are you willing to pay any amount to play this game given that in theory you can win upto Rs $\infty?$ \pause Can you see why this seems to be a paradox?
\end{alertblock}
\end{frame}



\begin{frame}{Analyzing Solution to Saint Petersberg Paradox Problem...}
\begin{alertblock}{Expected Value for Saint Petersberg Paradox...}
\vspace{-0.5cm}
\begin{align*}
E[X] &= 1 \times \dfrac{1}{2} + 4 \times \dfrac{1}{4} + 8 \times \dfrac{1}{8} \dots = 1 + 1 + 1 + \dots = \infty
\end{align*} \pause 
So, are you willing to pay any amount to play this game given that in theory (on average) you can win upto Rs $\infty?$ \pause Can you see why this seems to be a paradox?
\end{alertblock}
\pause 
%\newline
\vspace{1cm}

It is \yellow{paradoxical} because of the following reasons: \pause 
\begin{itemize}
\item In reality, very few would pay even Rs 15 to play this game. \pause \yellow{Why?} \pause 
\item \yellow{Paradox:} you should be willing to pay any amount, \pause but you \yellow{won't} be willing to do so!
\end{itemize}

\end{frame}


%

\begin{frame}{Analyzing Solution to Saint Petersberg Paradox Problem...}
\pause 
\begin{columns}
\begin{column}{0.5\textwidth}
\vspace{-0.8cm}
\begin{table}
\begin{tabular}{lllll}
$n$ & Prize & Prob & Utility & Exp. \\ \hline 
1 & Rs 2 & 1/2 & Rs 2 & 1 \\
2 & Rs 4 & 1/4 & Rs 4 & 1 \\
3 & Rs 8 & 1/8 & Rs 8 & 1 \\
4 & Rs 16 & 1/16 & Rs 16 & 1 \\
5 & Rs 32 & 1/32 & Rs 32 & 1 \\
6 & Rs 64 & 1/64 & Rs 32 & 0.5 \\
7 & Rs 128 & 1/128 & Rs 32 & 0.25 \\
8 & Rs 252 & 1/252 & Rs 32 & 0.125 \\
\vdots & \vdots & \vdots & \vdots & \vdots \\
\end{tabular}
\end{table}
\end{column} 

\pause 

\begin{column}{0.5\textwidth}
Let us make some observations. \pause 
\begin{itemize}
\item If we pay Rs 1, then if the game ends in 1st toss, then we get Rs 2. That is we gain Rs 1. \pause  
\item If we pay say Rs 10, and the game ends in 32nd toss, then on average, we get Rs 6, and we loose Rs 2. \pause 
\item Naturally, we wonder what amount we should be willing to pay? 
\end{itemize}
\end{column}
\end{columns} \pause 
\begin{alertblock}{Expected Value for Saint Petersberg Paradox...}
\vspace{-0.5cm}
\begin{align*}
E[X] &= 1 \times \dfrac{1}{2} + 4 \times \dfrac{1}{4} + 8 \times \dfrac{1}{8} \dots = 1 + 1 + 1 + 1 + 1 + 0.5 + 0.25 + \dots \approx 6
\end{align*} \pause 
So, we should not pay more than Rs 6 or much lower than this!
\end{alertblock}
\end{frame}

