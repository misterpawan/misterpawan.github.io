% \section{Motivation for Probability}

% %\begin{comment}

% \begin{frame}{Why do we need Probability?}
% \begin{alertblock}{What is probability?}
% Probability is a \yellow{measure} that we use to measure how likely an event or outcome is.
% For example, today there is 10\% chance of rain.
% \end{alertblock} \pause 

% \vspace{0.5 cm}
% How likely is that one player or team will win the game? \\

% \begin{figure}
% \includegraphics[scale=1.8]{ludo11}
% \end{figure}
% \end{frame}

% \begin{frame}{Motivation for Probability...}

% \begin{columns}

% \begin{column}{0.5\textwidth}
% How likely is that one will win the lottery?
% \end{column}

% \begin{column}{0.5\textwidth}
% \begin{figure}
% \includegraphics[scale=0.03  ]{lottery}
% \end{figure}
% \end{column}
% \end{columns}

% \pause 

% \begin{columns}
% \begin{column}{0.5\textwidth}
% How likely it is to rain or snow?
% \end{column}

% \begin{column}{0.5\textwidth}
% \begin{figure}
% \includegraphics[scale=0.15]{weather}
% \end{figure}
% \end{column}
% \end{columns}

% \end{frame}

% \begin{frame}{Study of Probability started with Games of Chance...}
% \begin{figure}
% \includegraphics[scale=0.45]{game11}
% \end{figure}
% \end{frame}

% \begin{frame}{Game of Chance, Probability, and War}

% \begin{columns}
% \begin{column}{0.65\textwidth}
% \begin{figure}
% \includegraphics[scale=0.3]{maha}
% \caption{illustration of great war}
% \end{figure}
% \end{column}

% \begin{column}{0.3\textwidth}
% \begin{itemize}
% \item Mahabharat, happened around 900 BCE (Proof dates to 400BCE) \pause 
% \item Struggle between two major group of cousins: Adventure, Drama, Suspense, Thriller, etc \pause 
% \item ... and a game of chance!
% \end{itemize}
% \end{column}

% \end{columns}

% \end{frame}


% \begin{frame}{What Does Game of Chance has to do with Mahabharat in India?}

% \begin{columns}
% \begin{column}{0.6\textwidth}
% \begin{figure}
% \includegraphics[scale=0.2]{pasha}
% \end{figure}
% \end{column}

% \begin{column}{0.4\textwidth}
% \begin{itemize}
% \item It is a game of Pasha, similar to modern ludo \pause
% \item In ludo, there is one dice with 6 sides 
% \item There are two cuboidal dices with number of dots on them \pause 
% \item The move is determined by a randomly throwing the two dices
% \end{itemize}
% \end{column}

% \end{columns}

% \end{frame}

% \begin{frame}{Game of Chance and Early History of Probability}

% \begin{columns}
% \begin{column}{0.35\textwidth}
% \begin{figure}
% \includegraphics[scale=0.4]{cardano}
% \caption{Cardino, Italy}
% \end{figure}
% \end{column}
% \pause 
% \begin{column}{0.7\textwidth}
% \begin{itemize}
% \item Early theory of probability arose from games of chance played in Europe\pause
% \item On the left is Cardano, an Italian mathematician, who studied game of chance \pause  
% \item He gambled for about 25 years! \pause 
% \item His work on probability were published in famous 15 page ``Liber de Ludo Aleae"
% \end{itemize}
% \end{column}

% \end{columns}

% \end{frame}


% \begin{frame}{Game of Chance and Early History of Probability...}

% \begin{columns}
% \begin{column}{0.3\textwidth}
% \begin{figure}
% \includegraphics[scale=0.4]{cardano}
% \end{figure}
% \end{column}
% \pause 
% \begin{column}{0.75\textwidth}
% \begin{itemize}
% \item Cardano’s contributions have led some to consider
% Cardano as the real father of probability. \pause
% \item Oystein Ore’s biography of Cardano,
% titled Cardano: The Gambling Scholar:
% \begin{figure}
% \includegraphics[scale=0.55]{cardano2}
% \end{figure}
% \end{itemize}
% \end{column}

% \end{columns}

% \end{frame}






%\begin{frame}{Definition of Classical Probability...}
%
%\begin{columns}
%\begin{column}{0.3\textwidth}
%\begin{figure}
%\includegraphics[scale=0.45]{laplace}
%\end{figure}
%\end{column}
%\pause 
%\begin{column}{0.75\textwidth}
%\begin{itemize}
%\item French mathematician, 1749-1827 \pause
%\item Known for his work in 
%\begin{itemize}
%\item Engineering: Tidal Dynamics \pause 
%\item Mathematics: Laplace equation, Laplace Transform \pause 
%\item Statistics: Bayesian Intrepretation \pause 
%\item Physics: Existance of black holes, Gravitational collapse, stability of solar, Speed of sound, Surface tension, etc
%\end{itemize}
%\end{itemize}
%\end{column}
%
%\end{columns}
%
%\end{frame}


% \begin{frame}%{Define Keywords}
% \begin{alertblock}{Random Experiment}
% A \yellow{random experiment} is a process by which we observe something \yellow{uncertain}. After the experiment, the result of the random experiment is known.
% \end{alertblock}

% \pause 

% \begin{alertblock}{Outcome}
% An \yellow{outcome} is a result of a \yellow{random} experiment. 
% \end{alertblock}

% \pause 

% \begin{alertblock}{Sample Space}
% The set of \yellow{all} possible outcomes is called the \yellow{sample space}.
% \end{alertblock}

% \pause 

% \begin{alertblock}{Event}
% An \yellow{event} is a \yellow{subset} of the sample space.
% \end{alertblock}

% \pause 

% \begin{alertblock}{Trial}
% When we repeat a random experiment several times, we call each one of them a \yellow{trial.}
% \end{alertblock}


% \end{frame}


% \begin{frame}{Examples of Experiments and Samples Spaces}
% \begin{itemize}
% \item \yellow{Random experiment:} toss a coin; \\ \pause 
% \yellow{Sample space:} S=\{heads,tails\} or as we usually write it, $\{H,T\}.$ \pause 
% \item \yellow{Random experiment:} roll a die;\\ \pause \yellow{Sample space:} S=\{1,2,3,4,5,6\}. \pause 
% \item \yellow{Random experiment:} observe the number of iPhones sold by an Apple store in Boston in 2015; \\ \pause \yellow{Sample space:} S=\{0,1,2,3, $\cdots$\}. \pause 
% \item \yellow{Random experiment:} observe the number of goals in a basketball match; \\ \pause \yellow{Sample space:} S=\{0,1,2,3,$\cdots$ \}. 
% \end{itemize}
% \end{frame}


% \begin{frame}{Definition of Classical Probability...}

% \begin{columns}
% \begin{column}{0.3\textwidth}
% \begin{figure}
% \includegraphics[scale=0.45]{cardano3}
% \end{figure}
% \end{column}
% \pause 
% \begin{column}{0.7\textwidth}
% \begin{itemize}
% \item In Chapter 14 of the De Ludo Aleae, Cardano gives what
% some would consider the first definition of classical (or
% mathematical) probability: \pause
% \begin{figure}
% \includegraphics[scale=0.58]{cardano4}
% \end{figure}
% \end{itemize}
% \end{column}

% \end{columns}

% \end{frame}


% \section{Classical Probability}



% \begin{frame}{Definition of Classical Probability...}
% \pause 
% \begin{columns}
% \begin{column}{0.3\textwidth}
% \begin{figure}
% \includegraphics[scale=0.45]{laplace}
% \caption{Pascal, France}
% \end{figure}
% \end{column}
% \pause 
% \begin{column}{0.75\textwidth}
% \begin{itemize}
% \item Pascal, French mathematician, 1749-1827 \pause
% \item Modern definition of probability is due to him \pause 
% \item Known for his work in 
% \begin{itemize}
% \item Engineering: Tidal Dynamics \pause 
% \item Mathematics: Laplace equation, Laplace Transform \pause 
% \item Statistics: Bayesian Intrepretation \pause 
% \item Physics: Existence of black holes, Gravitational collapse, stability of solar, Speed of sound, Surface tension, etc
% \end{itemize}
% \end{itemize}
% \end{column}

% \end{columns}

% \end{frame}


% \begin{frame}{Definition of Classical Probability...the way it evolved}
% \pause 
% \begin{figure}
% \includegraphics[scale=0.6]{laplace2}
% \caption{Book: Théorie analytique des probabilités, Laplace, 1812}
% \end{figure}
% \pause

% \alert{Translation of last line:} The probability of an event is a \yellow{fraction} whose \yellow{numerator is  
% the number of favourable cases} and whose \yellow{denominator is total number of cases.} In the first line, he also mentions that all the events are \yellow{equally possible} (or equally likely).

% \end{frame}


% \begin{frame}{Finite Probability Space}
% \begin{alertblock}{Definition}
% \begin{itemize}
% \item The number of outcomes is finite. \pause 
% \item Each outcome $i$ has probability $p_i \geq 0$ \pause 
% \item Probabilities sum to 1: $\sum_i p_i = 1$ \pause 
% \item \yellow{Event:} a set of outcomes \pause 
% \item \yellow{Probability of the event:} sum of outcome probabilities \pause 
% \item Is $p_i=0$ possible?
% \end{itemize}
% \end{alertblock}
% \end{frame}


% \begin{frame}{Finite Probability Space with Equally Likely Outcomes}
% \begin{alertblock}{Definition}
% A finite sample space where each outcome is equally likely
% \end{alertblock}
% \pause 
% \begin{itemize}
% \item Let $S = \{ s_1, s_2, \cdots, s_N \}, \quad P(s_i) = P(s_j), \quad \forall i,j$ \pause 
% \item Since all outcomes are equally likely, 
% \[ P(s_i) = \dfrac{1}{N}, \quad \forall i \in \{ 1,2, \cdots, N\} \] \pause 
% \item If $A$ is any event with cardinality $|A|=M,$ then \[ P(A) = \dfrac{|A|}{|S|} = \dfrac{M}{N}\]
% \end{itemize}
% \end{frame}



% \begin{frame}{Probability and Axioms of Probability}
% \pause 
% \begin{alertblock}{Probability}
% We assign a \yellow{probability measure} P(A) or Pr(A) to an event A. \pause This is a value between 0 and 1 that shows how likely the event is. \pause If P(A) is close to 0, it is very unlikely that the event A occurs. \pause On the other hand, if $P(A)$ is close to 1, $A$ is very likely to occur. \pause The main subject of probability theory is to develop tools and techniques to calculate probabilities of different events.
% \end{alertblock}
% \pause 
% \begin{alertblock}{Probability Axioms...}
% \begin{itemize}
% \item \yellow{Axiom 1:} For any event $A, \quad P(A) \geq 0$ \pause 
% \item \yellow{Axiom 2:} Probability of the sample space $S$ is $P(S)=1$ \pause 
% \item \yellow{Axiom 3:} If $A1,A2,A3,\cdots$ are disjoint events, then $P(A_1 \cup A_2 \cup A_3 \cdots ) = P(A_1)+P(A_2)+P(A_3)+ \cdots $
% \end{itemize}
% \end{alertblock}
% \end{frame}

% \begin{frame}{Quiz}
% \begin{alertblock}{Quiz}
% Prove the following:
% \begin{itemize}
% \item For any event $A,$ $P(A^c) = 1 - P(A)$ \pause 
% \item The probability of the empty set is zero, i.e., $P(\phi) = 0$ \pause 
% \item For any event $A, P(A ) \leq 1$ \pause 
% \item If $A \subseteq B,$ then $P(A) \leq P(B)$ \pause 
% \item $P(A - B) = P(A) - P(A \cap B)$ 
% \end{itemize}
% \end{alertblock}
% \end{frame}


% \begin{frame}{Calculate Probabilities...}
% \begin{alertblock}{Coin Toss example}
% A coin is considered fair if the likelihood of getting heads or tails is same. Calculate the probability of obtaining head, when the coin is tossed once.
% \end{alertblock}
% \pause 
% \begin{block}{Answer}
% There are two favourable cases, the set of all possible cases is $\{H, T \}.$ The number of head possible in one toss of the coin is 1. Hence, the probability of obtaining head is the fraction: $\dfrac{1}{2}$
% \end{block}

% \end{frame}

% \begin{frame}{Quiz}
% \begin{alertblock}{Quiz}
% We roll a fair dice. What is the probability of the event $E = \{ 1, 6\}?$
% \end{alertblock}
% \vspace{4cm}
% \end{frame}


% \begin{frame}{Calculate Probabilities...}
% \begin{alertblock}{Coin Toss example}
% A coin is considered \yellow{fair} if the likelihood of getting heads or tails is \yellow{same.} \pause  Calculate the probability of obtaining at least one head, when two fair coins are tossed simultaneously.
% \end{alertblock}
% \pause 
% \begin{block}{Answer}
% \begin{itemize}
% \item Total number possibilities are 

% $ \{ H, H \}, \{ H, T \}, \{ T, T\}, \{ T, H \} $ \pause 

% \item Favourable cases are
% %\begin{align*}
% $ \{ H, H \}, \{ H, T \}, \{ T, H \} $ \pause 
% %\end{align*}
% \item Hence the probability of atleast one head is = $\dfrac{3}{4}$
% \end{itemize}
% \end{block}

% \end{frame}


% \begin{frame}{Calculate Probabilities...}
% \begin{alertblock}{Coin Toss example}
% Ten \yellow{fair coins} are tossed \yellow{simultaneously}. What is the probability of getting \yellow{atleast one head?}
% \end{alertblock}
% \pause 
% \begin{block}{Answer}
% \begin{itemize}
% \item Total number possibilities are $2^{10}$ \pause 
% \item Only case with \yellow{no head:} $\{ T,T,T,T,T,T,T,T,T,T\}.$ \yellow{Favourable cases} are: $2^{10}-1$ \pause 
% \item Hence the probability of \yellow{atleast one head} is = $\dfrac{2^{10}-1}{2^{10}}$
% \end{itemize}
% \end{block}

% \end{frame}






% \begin{frame}{Computing probabilities...}

% \begin{columns}
% \begin{column}{0.45\textwidth}
% Consider an \yellow{experiment} of throwing two dice: blue and red one.
% \begin{figure}
% \includegraphics[scale=2.5]{redblue2}
% \end{figure}
% \end{column}
% \pause 

% \begin{column}{0.6\textwidth}
% What did we assume? \pause 
% \begin{itemize}
% \item The \yellow{outcome} is $(x,y), \quad x,y, \in \{1,2,\dots,6\}$ \pause 
% \item The \yellow{number of outcomes} = 36 \pause 
% \item We assume that all 36 are \yellow{equiprobable} \pause 
% \item \yellow{Probability space:} all outcomes \pause 
% \item \yellow{Event:} some favourable outcomes that we need to define precisely
% \end{itemize}
% \end{column}

% \end{columns}

% \end{frame}

%%%%%%%%%%%%%%%% QUIZ %%%%%%%%%%%%%%%%%%%%%%%
% \begin{frame}{Attendance Quiz-2}
% \Huge
% \url{https://tinyurl.com/y3ct7ucj}
% \normalsize
% \vspace{3cm}
% \end{frame}



\begin{frame}{Computing probabilities...}

\begin{columns}
\begin{column}{0.5\textwidth}
\yellow{Experiment:} Throw two dice. \\ \pause 
\yellow{Event:} Bigger number on red than blue. \\ \pause 
\yellow{Quiz:} What is the probability of this event? \pause 
\begin{figure}
\includegraphics[scale=0.5]{twodice2}
\end{figure}
\end{column}
\pause 

\begin{column}{0.5\textwidth}
\begin{itemize}
\item \yellow{Probability space:} all outcomes \pause 
\item \yellow{Event:} some favourable outcomes \pause 
\item \yellow{Probability:} (\#favorable)/36
\begin{figure}
\includegraphics[scale=0.5]{redblue3}
\end{figure}
\end{itemize}
\end{column}
\end{columns}
\pause 
\mydef{0.4}{Hence, probability, $p = \dfrac{15}{36}$}

\end{frame}


\begin{frame}
\begin{alertblock}{Quiz}
\begin{itemize}
\item \yellow{Experiment:} Toss a coin $n$ times \pause 
\item \yellow{Outcome:} sequence of $n$ bits \pause 
\item \yellow{Total number of outcomes:} $2^n$ outcomes \pause 
\item \yellow{Assumption:} equiprobable \pause 
\item \yellow{Probability of all heads:} $1/2^n$ \pause 
\item Consider an event of ``first bit" = ``last bit" \pause 
\item What is the \yellow{probability} of the above event?
\end{itemize}
\end{alertblock}
\end{frame}



\begin{frame}{Answer to Quiz...}
\pause 
\begin{block}{Answer to Quiz:}
\pause 
\begin{itemize}
\item Number of possible cases:
\begin{align*}
&H \, * \, \cdots \, * \, H \\
&T \, * \, \cdots \, * \, H \\
&H \, * \, \cdots \, * \, T \\
&T \, * \, \cdots \, * \, T \\
\end{align*} \pause 
\item \yellow{Observation:} In \yellow{half} of the cases, first bit does not equal last bit \pause 
\item Hence, probability of the event is $1/2$
\end{itemize}
\end{block}
\end{frame}

\begin{frame}{Remark on Counting. Does probability boils down to counting always?}
\pause 
\begin{itemize}
\item It is true that in many cases probability boils down to counting \pause 
\item However, there are \yellow{non-uniform} distributions  \pause 
\item We also study \yellow{continuous} distributions \pause 
\item In the continuous case, we need to consider some other measure \pause 
\item What is the probability that a number 1.5 is picked in $[1,2]?$ \pause Does this make sense? 	
\end{itemize}
\end{frame}


\begin{frame}{Probability, Galton Board, and Pascal Triangle}

\begin{center}
Movie of Galton board here!
\end{center}

\end{frame}


\begin{frame}{Probability, Galton Board, and Pascal Triangle}

\begin{columns}
\begin{column}{0.3\textwidth}
\begin{figure}
\includegraphics[scale=0.5]{galton}
\end{figure}
\end{column}
\pause 
\begin{column}{0.75\textwidth}
\begin{itemize}
\item This is called galton board, \yellow{Sir Francis Galton (1822-1911)} \pause
\item The set of red marbles are allowed to fall \pause 
\item We observe the distribution of the marbles after the fall \pause 
\item \yellow{Quiz:} Why do we have more marbles in the middle? \pause 
\item Does probability have any role in this distribution? \pause 
\item Can we \yellow{prove that the distribution will look like this mathematically?} \pause 
\item Does \yellow{Pascal triangle} has anything to do with this? \pause 
\item Let us analyze this in detail ...
\end{itemize}
\end{column}

\end{columns}

\end{frame}




%\begin{frame}{Probability, Galton Board, and Pascal Triangle}
%
%\begin{columns}
%\begin{column}{0.35\textwidth}
%\begin{figure}
%\includegraphics[scale=0.5]{galton2}
%\end{figure}
%\end{column}
%\pause 
%\begin{column}{0.7\textwidth}
%\begin{itemize}
%\item Assume that at each level the beans are splitted evenly. \pause
%\item The set of red marbles are allowed to fall
%\item We observe the distribution of the marbles 
%\item Why do we have more marbles in the middle?
%\item Can we prove this mathematically?
%\item Also, what does Pascal triangle has to do with this?
%\end{itemize}
%\end{column}
%
%\end{columns}
%
%\end{frame}





\begin{frame}{Probability, Galton Board, and Pascal Triangle}

\begin{columns}
\begin{column}{0.35\textwidth}
\begin{figure}
\includegraphics[scale=0.5]{galton2}
\end{figure}
\end{column}
\pause 
\begin{column}{0.7\textwidth}
\begin{itemize}
\item Assume that at each level the \yellow{beams are splitted evenly} \pause
\item Probability that ball hits current beam is sum of probability of this ball hitting left and right beam above \pause 
\item If $x$ is the top left probability, and $y$ is the top right probability, then probability of hitting the current bin is $(x+y)/2$ \pause 
\begin{figure}
\includegraphics[scale=0.35]{galton3}
\end{figure}
\pause 
\item The probability of current bin = $\binom{n}{k} / 2^n$
\end{itemize}
\end{column}

\end{columns}

\end{frame}


%\begin{frame}{Probability, Galton Board, and Pascal Triangle}
%
%\begin{columns}
%\begin{column}{0.3\textwidth}
%\begin{figure}
%\includegraphics[scale=0.5]{galton}
%\end{figure}
%\end{column}
%\pause 
%
%\begin{column}{0.75\textwidth}
%What did we assume?
%\begin{itemize}
%\item At each level half of the beans go left and half of the beans go right \pause
%\item Nothing but the truth, but not the entire truth \pause 
%\item Imagine that beans remember left/right direction, and it influences the next move
%\item May happen in real life; But we assumed \yellow{independence} \pause 
%\item We define independence later
%\end{itemize}
%\end{column}
%
%\end{columns}
%
%\end{frame}




\begin{frame}{Non equiprobable outcomes!}

\begin{columns}
\begin{column}{0.5\textwidth}
What is wrong with these dice?
\begin{figure}
\includegraphics[scale=0.55]{skewdice}
\end{figure}
\end{column}
\pause 

\begin{column}{0.5\textwidth}
\begin{itemize}
\item So far we assumed \yellow{equiprobable} outcomes \pause 
\item On the left, we have \yellow{skewed dice, i.e., unfair dice} \pause 
\item Will our previous definition of calculating probability work? \pause 
\item Unfortunately not, how to deal with skewed cases?

\end{itemize}
\end{column}
\end{columns}

\end{frame}


\begin{frame}{Non equiprobable outcomes!}

\begin{columns}
\begin{column}{0.35\textwidth}

\begin{figure}
\includegraphics[scale=0.55]{skewdice}
\end{figure}
\end{column}
\pause 

\begin{column}{0.65\textwidth}
\begin{itemize}
\item Since the dice is skewed, frequency of outcomes are \yellow{different} \pause
\begin{itemize}
\item That is, for example, number 4 may appear more frequently  \pause 
\end{itemize}
\item Let $p_i$ denote the probability of $i$th number \pause 
\item If we throw the dice long enough, the frequencies \yellow{stabilise} \pause 
\item For example, the probability of getting an even number = $p_2 + p_4 + p_6$ \pause 
\item Also, we know that the sum of \yellow{all the probabilities sum to 1, even though the dice is skewed!}
\begin{itemize}
\item That is: $p_1 + p_2 \cdots + p_6 = 1$
\end{itemize}

\end{itemize}
\end{column}
\end{columns}

\end{frame}






\begin{frame}{Example: Dice Throw and Venn Diagram}

\begin{columns}
\begin{column}{0.5\textwidth}
\yellow{Experiment:} Throw a fair dice once \\ \pause 
\yellow{Sample Space:} $S = \{ 1,2,3,4,5,6 \}$ \pause  
\begin{figure}
\includegraphics[scale=0.3]{probdice}
\end{figure}
\begin{itemize}
\item Consider \yellow{event} $A = \{1,2,3\}$ \pause 
\item Consider \yellow{event} $B = \{5,6\}$
\end{itemize}
\end{column}
\pause 

\begin{column}{0.5\textwidth}
\begin{itemize}
\item Since the dice is fair, we have $p_1+p_2+\cdots + p_6 = 1$ \pause 
\item $Pr(A) = p_1 + p_2 + p_3$ \pause 
\item $Pr(B) = p_5 + p_6$ \pause 
\item Consider the \yellow{event} $A$ or $B = \{ 1,2,3,5,6\}$ \pause 
\item $Pr(A ~\text{or}~ B) = $
\[ Pr(A) + Pr(B) = p_1+p_2+p_3+p_5+p_6 \]
\end{itemize}
\yellow{Why $Pr(A~\text{or}~B) = Pr(A)+Pr(B)?$} \pause \\
\yellow{When are we allowed to add probabilities?}
\end{column}
\end{columns}

\end{frame}


\begin{frame}{Mutually Exclusive Events Events}
\begin{alertblock}{Mutually Exclusive Events}
Two events are said to be \yellow{mutually exclusive} if they have disjoint set of outcomes. \pause Let $A$ and $B$ be two \yellow{mutually exclusive events}, then we have \[ Pr(A \cup B) = Pr(A) + Pr(B) \]
\end{alertblock}
\pause
\begin{alertblock}{Probability of a complement}
Let $A$ be any event, and $A^c$ be its \yellow{complement}. \pause Then clearly $A \cup A^c$ is the \yellow{complete set of outcomes,} i.e., it is a \yellow{sample space}. \pause  Hence, we have \[ Pr(A) + Pr(A^c) = 1 \quad \implies Pr(A^c) = 1 - Pr(A) \]
\end{alertblock}


\end{frame}

\begin{frame}{Mutually Exclusive Events..}
\pause 
\begin{alertblock}{Mutually disjoint events}
A set of $n$ events denoted by $A_1, A_2, \cdots, A_n$ are called \yellow{mutually disjoint events} if the following holds:
\pause 
\begin{align*}
P(A_1 \cup \cdots \cup A_n) = P(A_1) + \cdots + P(A_n).
\end{align*}
\pause 
From \yellow{Axiom-3,} it holds for \yellow{countably infinite unions}. 
\end{alertblock}
\end{frame}


\begin{frame}{Quiz: Use concept of disjoint events}
\begin{alertblock}{Quiz}
In a cricket tournament with four teams denoted by $\{ A, B, C, D \},$ team $A$ has $20\%$ chance of winning, while team $B$ has a $40\%$ chance of winning. What is the probability that $A$ or $B$ win the tournament? 
\end{alertblock}
\end{frame}

\begin{frame}{Quiz-Use probability axioms}
\begin{alertblock}{Quiz}
\yellow{Suppose we know the following:}
\begin{itemize}
\item there is a $50\%$ chance that it will be hot Today
\item there is a $30\%$ chance that it will be hot Tomorrow
\item there is $10\%$ chance that it wont be hot Today or Tomorrow 
\end{itemize} \pause 
\yellow{Answer the following:} \pause 
\begin{itemize}
\item probability that it will be hot today or tomorrow
\item probability that it will be hot today and tomorrow
\item probability that it will be hot today but not tomorrow
\item probability that it either will be hot today or tomorrow, but not both
\end{itemize}
\end{alertblock}
\end{frame}


\begin{frame}{Non-Mutually Exclusive Events}
\begin{itemize}
\item What if the two events $A$ and $B$ are \yellow{not mutually exclusive}? \pause 
\item What relation we have
\begin{align}
\label{eq1}
Pr(A \cup B) \quad ? \quad Pr(A) + Pr(B)
\end{align} \pause 
\item Let us consider an example below with two events $A$ and $B$ shaded
\begin{figure}
\includegraphics[scale=0.3]{excl}
\end{figure} \pause 
\item For the events $A$ and $B,$ which relation \eqref{eq1} holds?
\end{itemize}
\end{frame}


\begin{frame}%{Non-Mutually Exclusive Events}
\begin{itemize}
\item Consider the example as before
\begin{figure}
\includegraphics[scale=0.3]{excl}
\end{figure}
\pause 
\item Let us calculate the probabilities of the two events $A$ and $B$
\[ Pr(A) = p_1 + p_2 + p_3, \quad Pr(B) = p_3 + p_4 \] \pause 
\item Now we calculate probability of $A \cup B$ \pause 
\[ Pr(A \cup B) = Pr(\{ 1,2,3,4 \}) = p_1 + p_2 + p_3 + p_4 \] \pause 
\item Using above: \yellow{$Pr(A \cup B) = Pr(A) + Pr(B) - Pr(A \cap B)$} \pause 
\item Recall discrete mathematics: \yellow{inclusion-exclusion principle}. (\yellow{Proof?})
\end{itemize}
\end{frame}

\begin{frame}{Generalized Inclusion-Exclusion Principle}
\pause 
\begin{alertblock}{Inclusion-Exclusion Principle}
\begin{itemize}
\item $P(A \cup B) = P(A) + P(B) - P(A \cap B)$ \pause 
\item $P(A \cup B \cup C) = P(A) + P(B) + P(C) - P(A \cap B) - P(B \cap C) - P(B \cap C) + P(A \cap B \cap C)$ \pause 
\item In general for $n$ events, we have \pause 
\begin{align*}
P(\cup_{i=1}^n A_i ) = \sum_{i=1}^n P(A_i) - \sum_{i < j} P(A_i \cap A_j) &+ \sum_{i<j<k} P(A_i \cap A_j \cap A_k) \\ &- \cdots (-1)^{n-1} P(\cap_{i=1}^n A_i)
\end{align*}
\end{itemize}
\end{alertblock}
\end{frame}

\begin{frame}{Discrete Probability Models}
\pause 
\begin{itemize}
\item Sample space can be of two types: \yellow{discrete or continuous} \pause 
\item in discrete probability models we can compute the probability of events by \yellow{adding} all the corresponding outcomes \pause 
\item in continuous probability models we need to use \yellow{integration instead of summation}	
\end{itemize} \pause 
\begin{alertblock}{Definition of discrete probability model}
If the sample space $S$ is a countable set, then it refers to discrete probability model. Since $S$ is countable: $S = \{ s_1, s_2, \cdots, \}$ \pause 
\begin{itemize}
\item For an event $A \subset S,$ by \yellow{3rd axiom}
\begin{align*}
P(A) = \sum_{s_j \in A} P(s_j)
\end{align*} 
\end{itemize}
\end{alertblock}
\end{frame}


\begin{frame}{Choices and Choice Tree}
\pause 
\begin{itemize}
\item Consider 6 balls labelled 1,2,3,4,5,6 \pause 
\item The balls are kept in two boxes. In the first box, we have ${1,2}$ balls, and in the second box, we have ${3,4,5,6}$ balls \pause 
\begin{figure}
\includegraphics[scale=0.4]{box}
\end{figure} \pause 
\item Assume the following: \pause 
\begin{itemize}
\item The two boxes are \yellow{equiprobable} \pause 
\item All the balls in each box is \yellow{equiprobable} \pause 
\end{itemize}
\item \yellow{Question:} What are the probabilities $p_1, p_2, \cdots, p_6?$
\end{itemize}	
\end{frame}


\begin{frame}
\begin{columns}

\begin{column}{0.4\textwidth}
\begin{figure}
\includegraphics[scale=0.37]{choice}
\caption{Consider a choice tree for the problem}
\end{figure}
\end{column}
\pause 
\begin{column}{0.65\textwidth}
\begin{itemize}
\item We have the following
$$ p_1 + p_2 = 1/2, \quad p_3 + p_4 + p_5 + p_6 = 1/2 $$ \pause 
\item Balls in box-1 are \yellow{equiprobable:} $p_1 = p_2$ \pause 
\item Balls in box-2 are \yellow{equiprobable:} $p_3 = p_4 = p_5 = p_6$ \pause 
\item From above, we have 
\[ p_1 = p_2 = 1/4, \quad p_3 = p_4 = p_5 = p_6 = 1/8 \]
\end{itemize}
\end{column}

\end{columns}
\end{frame}


%\end{comment}

\begin{frame}{Shrewd Prisoner and Maximizing Chance!}
\begin{alertblock}{Shrewd Prisoner Problem}
Long ago a prisoner was to be executed. \pause In
response to his supplications, he was promised that he would be
released if he drew a white ball from one of two similar urns. \pause  The
provisions were that he had to distribute 50 white and 50 black balls
between the two urns, \pause in any way he liked, \pause  after which he had to draw
a ball at random from one of these urns.
\\
\pause 
\yellow{Question:} How should the prisoner put the balls such that the probability of his release is maximized?
\end{alertblock}
\end{frame}



\begin{frame}{Solution to Shrewd Prisoner Problem}
\begin{itemize}
\item How did the prisoner arrange to make his chance of success as great
as possible? \pause 
\item What happens if he puts equal number of white and black balls in two boxes? \pause 
\begin{itemize}
\item The probability of choosing a white ball (from either boxes) is $1/2$ \pause 
\item Can he do better? \pause 
\end{itemize}
\item What if he puts all white in one, and all black in other urn? \pause 
\begin{itemize}
\item probability of choosing white from urn-1 is 1/2, but probability of choosing white from urn-2 is 0. Can he do better? \pause 
\end{itemize}
\item Can he do better than this? If yes, then how? \pause 
\item Put 1 white and 0 black in urn-1, and put 49 white and 100 black in urn-2 \pause 
\item The \yellow{probability of success} is $\dfrac{1}{2} \times 1 + \dfrac{1}{2} \times \dfrac{49}{99} \approx 3/4$
%\item If he chooses urn-1, he is sure, if not, then he still has a ``good" chance}
\end{itemize}
\end{frame}

\begin{frame}{Extension of the problem}
\begin{alertblock}{Quiz-1}
What is the chance of success of the prisoner if he had been allowed to distribute 100 balls among 4 urns?
\end{alertblock}
\pause 
\begin{alertblock}{Quiz-2}
What if the number of balls is increased?
\end{alertblock}

\end{frame}


%\begin{frame}{Random Walks, Independence, and Probability}
%\pause
%	
%\end{frame}























