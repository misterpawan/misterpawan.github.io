\section{Tuples}

\begin{frame}{Tuples}
\begin{alertblock}{Question}
Suppose we have a set of $n$ symbols. How many different
sequences of length $k$ we can form out of these symbols?
\end{alertblock}
\begin{itemize}
\item can apply the same argument as above! (product rule!) \pause 
\item there are $n$ possibilities to pick the first letter \pause 
\item similarly, There are $n$ possibilities to pick the second letter, and so on... \pause 
\item thus, the answer is a product of $n$ by itself $k$ times, that
is $n^k$
\end{itemize}
\end{frame}



\begin{frame}{Number of Number Plates}
\pause 
Consider the typical vehicle number plate in India \pause 
\begin{figure}
\includegraphics[scale=0.35]{plate2}
\end{figure} \pause 
\begin{itemize}
\item the first two letters denote state \pause 
\item the following two-digit number stands for district number \pause 
\item the following two-letter is RTO series \pause 
\item the last four digits is the vehicle number ranging from 0000 to 9999
\end{itemize} \pause 
\begin{alertblock}{Question}
How many vehicles are there? 
\end{alertblock}
\end{frame}


\begin{frame}{Tuples with Restrictions (Combine Sum and Product Rule)} 
\pause 
\begin{alertblock}{Question}
How many integer numbers are there between 0 and 9999 that have exactly one 
5 digit?
\end{alertblock} \pause
\begin{itemize}
\item Numbers between 0 and 9999 are sequences
of digits of length 4 \pause 
\item Three digital numbers correspond to
sequences starting with 0 \pause 
\item We can place the unique 5 at any of four positions \pause 
\item This gives us 4 cases; if we compute the number of
sequences in all four cases, we can get the answer by the
rule of sum \pause 
\item If we fix 5 in one place, then there are $5 \times 5 \times 5 = 125$ sequences \pause 
\item There are 4 ways to arrange 5 among 4 places \pause 
\item Hence, there are $4 \times 125 = 500$ four digit numbers below 10,000 with exactly one 5 
\end{itemize}
\end{frame}

\section*{Permutations}
\begin{frame}{Permutations}
\pause 
\begin{alertblock}{Definition}
Tuples of length $k$ without repetitions are called
\yellow{$k-$permutations}
\end{alertblock}
\pause 
\begin{alertblock}{Question}
\pause 
Suppose we have a set of $n$ symbols. How many different
sequences of length $k$ we can form out of these symbols if
we are not allowed to use the same symbol twice?
\end{alertblock}
\pause 
\begin{itemize}
\item Observe that if $n<k$ then there are no
$k-$permutations: there are simply not enough different
letters \pause 
\item So it is enough to solve the problem for the case $k \leq n$
\end{itemize} \pause 
\begin{figure}
\includegraphics[scale=0.3]{perm}
\end{figure} \pause 
\begin{itemize}
\item Hence there are 
%\begin{align*}
\[ n \times (n-k) \times \cdots (n-k+1) \]
%\end{align*} 
$k-$permutations, which is \yellow{$n! / (n-k)!$}
\end{itemize}
\end{frame}

\begin{frame}{Permutation Examples}
\pause 
\begin{alertblock}{Question}
In how many ways we can arrange $n$ different books in $n$ different bins on shelf?
\end{alertblock}
\pause 
\begin{alertblock}{Answer}
Hint: Use previous result with $k=n.$ 
\end{alertblock}
\end{frame}