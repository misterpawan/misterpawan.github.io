\section{Random Walks}

% \begin{frame}{Random Walks, Choice Trees, and Probability}
% \begin{center}
% Movie of 1st Brownian Motion
% \end{center}
% \end{frame}

% \begin{frame}{Random Walks, Choice Trees, and Probability}
% \begin{center}
% Movie of 2nd Brownian Motion
% \end{center}
% \end{frame}

\begin{frame}{Random Walk, Probability}
\begin{figure}
\includegraphics[scale=0.25]{randWalk2}
\end{figure} \pause 
\begin{itemize}
\item Consider a person at $x=0,$ he can travel one step to the right or to the left \pause 
\begin{itemize}
\item He can travel one step to the right with probability $q$ \pause 
\item He can travel one step to the left with probability $(1-q)$
\end{itemize}
\end{itemize} \pause 
\begin{alertblock}{Simple Random Walk in 1D}
A walk is called \yellow{simple random walk} in 1D if there is a \yellow{equal} probability of either going to right or going to the left. Above, we set $p=q=1/2$
\end{alertblock} \pause 
\begin{alertblock}{Question}
What is the probability that the person after $i$th step is at $x=0?$
\end{alertblock}
\end{frame}

\begin{frame}{Analysis of Simple Random Walk in 1D}
\pause 
\begin{figure}
\includegraphics[scale=0.25]{randWalk2}
\end{figure} \pause 
Draw the choice tree (Hint: Galton Board!): \\
\vspace{4cm}
\end{frame}

\begin{frame}{Analysis of Biased Random Walk in 1D, Binomial Distribution}
\pause 
\begin{figure}
\includegraphics[scale=0.25]{randWalk2}
\end{figure} \pause 
Draw the choice tree for \yellow{unbiased random walk, derive binomial distribution:} \\
\vspace{4cm}
\end{frame}


\begin{frame}{Random Walkers on 2D Grid...}
\pause 
\begin{columns}
\begin{column}{0.3\textwidth}
\begin{figure}
\includegraphics[scale=0.2]{randWalk2D1}
\end{figure}
\end{column}

\begin{column}{0.7\textwidth}
\begin{itemize}
\item Alice starts at point A, and each second, walks one edge right or up (if a point has two options, each direction has a 50% chance) until Alice reaches point B. 
\pause 
\item At the same time, Bob starts at point B, and each second he walks one edge left or down (if a point has two options, each direction has a 50% chance) in order to reach point A. 
\pause 
\item What is the probability Alice and Bob meet during their random walks?
\end{itemize}
\end{column}
\end{columns}
\end{frame}



\begin{frame}{Solution: Random Walkers on 2D Grid...}
\pause 
\begin{columns}
\begin{column}{0.3\textwidth}
\begin{figure}
\includegraphics[scale=0.2]{randWalk2D4}
\end{figure}
\end{column}

\begin{column}{0.7\textwidth}
\begin{itemize}
\item If Alice and Bob meet, they can meet only at the blue circles, which is after both have taken 5 steps
\pause 
\item Number of ways they could have taken the 5 steps is $2^5$ each, so combined, by product rule they can take $2^5 2^5 = 4^5 = 1024$ total paths!
\pause 
\item The number of ways Bob can reach the blue dots is given by binomial coefficients or Pascal's triangle! Same for Alice \pause 
\begin{itemize}
\item Total ways Bob and Alice could meet at blue dots is square of binomial cofficients
\end{itemize} \pause 
\item Hence, the total number of ways Bob and Alice could meet
\[ 1^2 + 5^2 + 10^2 + 5^2 + 1^2 = 252 \]
\item The probability that they meet is 
\[ 252/1024 = 24.6\% \]
\end{itemize}
\end{column}
\end{columns}
\end{frame}




%\end{comment}
















































