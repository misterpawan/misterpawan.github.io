\section{Problems}

%\begin{comment}
%
\begin{frame}{Distributing Assignments Among People}

\pause 

\begin{alertblock}{Problem}
Suppose there are 4 people and 9 different assignments. Each person should receive one assignment. Assignments for different people should be different. How many ways are there to do it?
\end{alertblock} 

\pause 

\begin{alertblock}{Quiz}
Where does this problem fit in our combination and permutation table?
\end{alertblock}

\end{frame}
%
%%\begin{comment}
%
\begin{frame}{Distributing Assignments Among People}

\pause 

\begin{alertblock}{Problem}
Suppose there are 4 people and 9 different assignments. Each person should receive one assignment. Assignments for different people should be different. How many ways are there to do it?
\end{alertblock} 

\pause 

\begin{block}{Finding the type of problem}
\begin{itemize}
\item This is a problem of selecting assignments for 4 people \pause 
\item Here assignments and people are different: so a ordered case \pause 
\item No repetitions: same assignment is not given to two persons \pause 
\item It is a case of ordered without repetitions, i.e., $k-$permutations
\end{itemize}
\end{block}
\end{frame}
%
%
\begin{frame}{Distributing Assignments Among People}

\pause 

\begin{alertblock}{Problem}
Suppose there are 4 people and 9 different assignments. Each person should receive one assignment. Assignments for different people should be different. How many ways are there to do it?
\end{alertblock} 

\pause 

\begin{block}{Answer}
\begin{itemize}
\item Let us have a look at the possibilities \pause 
\begin{tabular}{|l|l|l|l|l|}
\hline 
Persons & 1 & 2 & 3 & 4 \\ \hline
Number of options & 9 & 8 & 7 & 6 \\ \hline
\end{tabular} \pause 
\item The answer is $\dbinom{9!}{(9-4)!} = 9 \times 8 \times 7 \times 6 = 3024$
\end{itemize}
\end{block}
\end{frame}
%
%
\begin{frame}{Distributing Assignments Among People: Twist}
\pause 

\begin{alertblock}{Problem with a twist!}
There are 4 people and 9 different assignments. We need to distribute all assignments among people. No assignment should be assigned to two people. Every person can be given arbitrary number of assignments from 0 to 9. How many ways are there to do it?
\end{alertblock}

\pause 

\begin{alertblock}{Quiz}
Which category this problem belongs to? \includegraphics[scale=0.02]{pondering}
\end{alertblock}

\end{frame}
%
\begin{frame}{Distributing Assignments Among People: Twist}
\pause 
\begin{alertblock}{Problem with a twist!}
There are 4 people and 9 different assignments. We need to distribute all assignments among people. No assignment should be assigned to two people. Every person can be given arbitrary number of assignments from 0 to 9. How many ways are there to do it?
\end{alertblock}
\pause 

\begin{block}{Attempt to analyze the problem and fit it into a known case}
\begin{itemize}
\item Each person receives several assignments; look at one by one \pause 
\item First person can be assigned arbitrary number of assignments \pause 
\begin{itemize}
\item It is a case of counting all possible subsets of assignments
\end{itemize} \pause 
\item Same assignment can't be given to two persons, the number of subsets of assignments for second person depends on what we chose for first person! \yellow{How to attack this problem?}
\end{itemize}
\end{block}

\end{frame}
%
%
%
\begin{frame}{Distributing Assignments Among People: Twist}
\pause 
\begin{alertblock}{Problem with a twist!}
There are 4 people and 9 different assignments. We need to distribute all assignments among people. No assignment should be assigned to two people. Every person can be given arbitrary number of assignments from 0 to 9. How many ways are there to do it?
\end{alertblock}

\pause 

\begin{block}{Attempt to solve the problem and fit it into a known case}
\begin{itemize}
\item \yellow{Idea:} What if you look from other point of view? \pause 
\item Rather than giving assignments to people, assign people to assignments \pause 
\end{itemize}
\end{block}

\end{frame}
%
%
%
\begin{frame}{Distributing Assignments Among People: Twist}
\pause 
\begin{alertblock}{Problem with a twist!}
There are 4 people and 9 different assignments. We need to distribute all assignments among people. No assignment should be assigned to two people. Every person can be given arbitrary number of assignments from 0 to 9. How many ways are there to do it?
\end{alertblock}
\pause 

\begin{block}{Attempt to solve the problem and fit it into a known case}
\begin{tabular}{|l|l|l|l|l|l|l|l|l|l|}
\hline
Assignments & 1 & 2 & 3 & 4 & 5 & 6 & 7 & 8 & 9 \\ \hline
Options & 4 & 4 & 4 & 4 & 4 & 4 & 4 & 4 & 4  \\ \hline
\end{tabular}
\pause 
\begin{itemize}
\item There are $4^9 = 262144$ choices! This was a case of \yellow{Tuples}!
\end{itemize}
\end{block}

\end{frame}
%
%\end{comment}



\begin{frame}{Quiz-1}
\begin{figure}
\includegraphics[scale=0.15]{Quiz-1}
\end{figure}
\Huge \quad https://tinyurl.com/y2g93ofb 
%https://tinyurl.com/y2g93ofb
\normalsize
\end{frame}


\begin{frame}{Distributing Candies Among Kids}
\pause 
\begin{alertblock}{Problem}
There are 15 identical candies. How many ways are there to distribute them among 7 kids?
\end{alertblock}
\pause 
\begin{figure}
\includegraphics[scale=0.4]{candy2}
\end{figure}
\end{frame}


\begin{frame}{Distributing Candies Among Kids}
\pause 
\begin{alertblock}{Problem}
There are 15 identical candies. How many ways are there to distribute them among 7 kids?
\end{alertblock}
\pause 
\begin{block}{Analyzing the problem, and fitting it into the known case}
\begin{itemize}
\item When distributing candies, we pick one of 7 kids \pause 
\item Since all the candies are same, \yellow{repetitions are allowed} \pause 
\item Since the candies to be distributed are identical, \yellow{ordering doesn't matter} \pause 
\item This is a case of \yellow{combinations with repetitions}
\end{itemize}

\end{block}
\end{frame}



\begin{frame}{Distributing Candies Among Kids: Solution Finally!}
\pause 
\begin{alertblock}{Problem}
There are 15 identical candies. How many ways are there to distribute them among 7 kids?
\end{alertblock}
\pause 
\begin{block}{Analyzing the problem, and fitting it into the known case}
\begin{itemize}
\item Number of combinations = size of combination \pause 
\item Number of options = Number of kids \pause 
\item The answer is $\dbinom{15+(7-1)}{(7-1)} = \dbinom{21}{3} = 54264$
\end{itemize}

\end{block}
\end{frame}

%\begin{comment}

\begin{frame}{Fair Distributions..}
\pause 
\begin{alertblock}{A problem with previous distribution...}
No kid will like not having even a single candy. Infact, a kid would like to have it all! How can we change the problem?
\end{alertblock}
\pause 
\begin{figure}
\includegraphics[scale=0.15]{candy3}
\end{figure}
\end{frame}


\begin{frame}{Fair Distributions..}
\pause 
\begin{block}{A problem with previous distribution...}
We want that each kid receives atleast one candy!
\end{block}
\pause 
\begin{figure}
\includegraphics[scale=0.3]{candy4}
\end{figure} \pause 
Let us change our previous problem so that each kid is ensured \yellow{atleast one} candy!
\end{frame}




\begin{frame}{Distributing Candies with Fair Distribution...}
\pause 
\begin{alertblock}{Question}
There are 15 identical candies. How many ways are there to distribute them among 7 kids in such a way that each kid receives \yellow{at least 1} candy?
\end{alertblock}
\pause 

\begin{alertblock}{Quiz}
\begin{itemize}
\item Does our previous approach work here? What should you do? \pause 
\item Can we reduce this problem to previous problem? If yes, how?
\end{itemize}
\end{alertblock}

\end{frame}


\begin{frame}{Distributing Candies with Fair Distribution...}
\pause 

\begin{alertblock}{Question}
There are 15 identical candies. How many ways are there to distribute them among 7 kids in such a way that each kid receives \yellow{at least 1} candy?
\end{alertblock}
\pause 

\begin{block}{Idea}
\begin{itemize}
\item Give kids atleast one candy. Then we are left with 15-7=8 candies... \pause 
\item Now we can distribute these 8 candies as in previous problem! \pause 
\item It becomes a problem of \yellow{Combinations with Repetitions} \pause 
\begin{itemize}
\item \#combinations = 8; \quad \#options = 7
\end{itemize} \pause 
\item Answer = $\dbinom{8+(7-1)}{(7-1)} = \dbinom{14}{6} = 3003$
\end{itemize}
\end{block}

\end{frame}


\begin{frame}{How many kids can have no candies?}
\pause 

\begin{alertblock}{Question}
Assume that 15 identical candies are distributed among 7 kids. How many ways are there that will leave some kids without any candy?
\end{alertblock}
\pause 

\begin{figure}
\includegraphics[scale=0.2]{candy5}
\end{figure}
\pause 

\begin{block}{Hint}
Use the answer to previous two problems!
\end{block}

\end{frame}


\begin{frame}{Numbers with Fixed Sum of Digits}
\pause 
\begin{alertblock}{Problem}
How many non-negative integer numbers are there below 10 000 such that their sum of digits is equal to 9?
\end{alertblock}
\pause 
\begin{block}{Understanding the problem...}
\begin{itemize}
\item It is clear that we have nine options for the 1st digit \pause 
\item But what next? How many options for the 2nd digit?  \pause 
\item It is already tricky... 
\end{itemize}
\end{block}

\end{frame}



\begin{frame}

\begin{alertblock}{Problem}
How many non-negative integer numbers are there below 10 000 such that their sum of digits is equal to 9?
\end{alertblock} \pause 

\begin{itemize}
\item There are \yellow{four} positions to fill \pause 
\[ \yellow{\ast \: \ast \: \ast \: \ast} \] \pause 
\item The \yellow{sum of the digits must be 9}. We can \yellow{start with all zeros!}
\[ \yellow{0 \: 0 \: 0 \: 0} \] \pause 
\end{itemize}

\end{frame}

%
%
\begin{frame}
\begin{alertblock}{Problem}
How many non-negative integer numbers are there below 10 000 such that their sum of digits is equal to 9?
\end{alertblock} \pause 
\begin{itemize}
\item Idea: How many ways we can distribute 9 ones in 4 places? \pause 
\item Multiple ones will go to at least one of the four places, and get added \pause 
\item This belongs to the category: ``unordered with repetitions" \pause 
\item \#combinations = 9; \quad \#options = 4  \pause 
\item Hence, answer is $\dbinom{9 + (4-1)}{(4-1)} = \dbinom{12}{3} = 220$
\end{itemize}
\end{frame}


%%\begin{comment}
%
%
\begin{frame}{Numbers with Fixed Sum of Digits: A Twist}
\pause 
\begin{alertblock}{Problem}
How many non-negative integer numbers are there below 10 000 such that their sum of digits is equal to 10?
\end{alertblock}
\pause 
\begin{itemize}
\item Looks very similar to the previous one; Distribute ten ones between four positions \pause 
\item Combinations of size 10 among 4 options \pause 
\item Hence, answer is $\dbinom{10+(4-1)}{(4-1)} = \dbinom{13}{3} = 286$ \pause 
\item \yellow{But this is wrong! Why? The answer is off by 4!}
\end{itemize}
\end{frame}

%
\begin{frame}{Numbers with Fixed Sum of Digits: A Twist}
\pause 
\begin{alertblock}{Problem}
How many non-negative integer numbers are there below 10 000 such that their sum of digits is equal to 10?
\end{alertblock} \pause 
\begin{itemize}
\item When we assign all 10 ones to one position, but digits can be only upto 9 \pause 
\item We need to substract the additional cases! How? \pause 
\item By our previous approach, we would have distributed errorenously at either of the four places of four digit number \pause 
\item Since there are 4 cases where we could have assigned 10, these are the extra cases \pause 
\item \yellow{Hence, final correct answer = 286-4 = 282}
\end{itemize}
\end{frame}

%
%
%
%
\begin{frame}{Numbers with Non-increasing Digits Problem}
\pause 
\begin{alertblock}{Problem}
How many four-digit numbers are there such that their
digits are not increasing? Three-digit numbers are also four-digit, they just start with 0
\end{alertblock} \pause 
\begin{itemize}
\item If we try to count options for each position and apply the product rule, there are problems \pause 
\item 10 options for the first position, but for the second the number of options depends on the first number \pause 
\item Usual approach doesn't work; try from another angle!
\end{itemize}
\end{frame}


%
%
\begin{frame}{Numbers with Non-increasing Digits Problem}
\pause 
\begin{alertblock}{Problem}
How many four-digit numbers are there such that their
digits are not increasing? Three-digit numbers are also four-digit, they just start with 0
\end{alertblock} \pause 
\begin{itemize}
\item We pick digits from 0 to 9 to be in our number \pause 
\item Once we picked four digits, our number is uniquely determined \pause 
\item Order of picks does not matter; repetitions are allowed \pause 
\item \#combinations = 4; \quad \#options = 10 \pause 
\item The answer is $\dbinom{4 + (10-1)}{(10-1)} = \dbinom{13}{4} = 715$
\end{itemize}
\end{frame}



%
\begin{frame}{Group Work}
\pause 
\begin{alertblock}{Fact}
Group learning is a wonderful way to learn and share knowledge.
\end{alertblock} 
\begin{figure}
\includegraphics[scale=0.09]{group}
\end{figure}
\end{frame}

\begin{frame}{Splitting into Working Groups}
\pause 
\begin{alertblock}{Problem}
There are 12 students in the class. How many ways are there to split them into working groups of size 2 to work on the same assignment?
\end{alertblock} \pause 
\begin{itemize}
\item This problems is more tricky; There are several ways to solve it \pause 
\item But we need to combine several ideas \pause 
\item How do we approach this problem?
\end{itemize}
\end{frame}
%
%
\begin{frame}{Splitting into Working Groups}
\pause 
\begin{alertblock}{Problem}
There are 12 students in the class. How many ways are there to split them into working groups of size 2 to work on the same assignment?
\end{alertblock} \pause 
\begin{itemize}
\item We have to pick 2 people out of 12 in a working group \pause 
\item Order in the group does not matter, so combinations, so there are $\dbinom{12}{2}$ ways to do it \pause 
\item For the second group we have 10 people left, there are $\dbinom{10}{2}$ options, so on \pause 
\item So, overall we have the following options
\[ \dbinom{12}{2} \times \dbinom{10}{2} \times \dbinom{8}{2} \times \dbinom{6}{2} \times \dbinom{4}{2} \times \dbinom{2}{2} \] \pause 
\item \yellow{Is this correct answer?} \pause \yellow{No! Why?}
\end{itemize}
\end{frame}


%
%
\begin{frame}
\begin{itemize}
\item Let us number the students...
\end{itemize} \pause 
\begin{figure}
\includegraphics[scale=0.35]{12-kids}
\end{figure} \pause 
\begin{itemize}
\item According to our enumeration, we could have the following
\[ \{ 3,7 \}, \{ 1,5 \}, \{ 6, 9 \}, \{ 11, 2 \}, \{ 8, 12 \}, \{ 4,10 \} \]
\[ \{ 1, 5 \}, \{ 3,7 \}, \{ 6, 9 \}, \{ 11,2 \}, \{ 8, 12\}, \{ 4, 10 \} \] \pause 
\item That is, ordering across groups also does not matter!
\end{itemize}
\end{frame}


%
%
\begin{frame}{Finally, the solution}
\begin{alertblock}{Our Problem}
There are 12 students in the class. How many ways are there to split them into working groups of size 2 to work on the same assignment?
\end{alertblock} \pause
\begin{itemize}
\item That is, ordering across groups also does not matter! \pause 
\item We had 6! redundant splitting, which we need to divide by \pause 
\item We have counted each assignment 6! times, hence, after dividing, \yellow{the answer is}
\[ \dbinom{12}{2} \times \dbinom{10}{2} \times \dbinom{8}{2} \times \dbinom{6}{2} \times \dbinom{4}{2} \times \dbinom{2}{2} \times \dfrac{1}{6!} = \dfrac{12!}{6! \times 2^6} = 10395 \]
\end{itemize}
\end{frame}


%
%
%%\end{comment}
%
%%\section*{Importance of Probability and Statistics}
%
%%\begin{frame}{Importance of Probability and Statistics}
%%
%%\end{frame}
%
%
%%\end{comment}
%
%





