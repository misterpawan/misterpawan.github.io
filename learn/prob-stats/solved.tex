\section{Solved Problems}

\begin{frame}{Problem-1}
\begin{alertblock}{Problem-1}
Consider the PDF of the random variable $X$ \pause 
\begin{align*}
f_X(x) = \begin{cases}
cx^2 \quad &|x| \leq 1 \\
0 \quad &\text{otherwise}
\end{cases}
\end{align*} \pause 
\begin{itemize}
\item Find the constant $c$ \pause 
\item Find $E[X]$ and $\text{Var}[X]$ \pause 
\item Find $P(X \geq \dfrac{1}{2})$
\end{itemize}
\end{alertblock}
\end{frame}



\begin{frame}{Solution to Problem-1}
\begin{columns}
\begin{column}{0.5\textwidth}
\begin{enumerate}
\item 
To find $c,$ we have 
\begin{align*}
1 &= \int_{-\infty}^{\infty} f_X(u) \, du \\
&= \int_{-1}^1 cu^2 \, du \\
&= \dfrac{2}{3} c \implies c = \dfrac{3}{2}.
\end{align*} \pause 
\item To find $E[X],$ we have 
\begin{align*}
E[X] &= \int_{-1}^1 u f_X(u) \, du \\
&= \dfrac{3}{2} \int_{-1}^1 u^3 \, du \\
&= 0
\end{align*}

\end{enumerate}
\end{column} \pause 
\begin{column}{0.5\textwidth}
\begin{enumerate}
\item We have 
\begin{align*}
\text{Var} &= E[X^2] - E[X]^2 \\
&= \int_{-1}^1 u^2 f_X(u) \, du \\
&= \dfrac{3}{2} \int_{-1}^1 u^4 \, du = \dfrac{3}{5}
\end{align*} \pause 
\item To find $P(X \geq \dfrac{1}{2}),$ we have 
\begin{align*}
P(X \geq \dfrac{1}{2}) = \dfrac{3}{2} \int_{1/2}^1 x^2 \, dx = \dfrac{7}{16}.
\end{align*}
\end{enumerate}
\end{column}
\end{columns}
\end{frame}


\begin{frame}{Problem-2}
\begin{alertblock}{Problem 2}
Consider the PDF of continuous random variable $X$ 
\begin{align*}
f_X(x) = \dfrac{1}{2} e^{-|x|}, \quad \text{for all}~x \in \mathbb{R}
\end{align*}
If $Y = X^2,$ find the CDF of $Y.$
\end{alertblock}
\end{frame}

\begin{frame}{Solution to Problem-2}
\begin{enumerate}
\item We have $R_Y = [0, \infty).$ \pause 
\item For $y \in [0, \infty),$ \pause 
\begin{align*}
F_Y(y) &= P(Y \leq y) = P(X^2 \leq y) \\
&= P(-\sqrt{y} \leq X \leq \sqrt{y}) \\
&= \int_{-\sqrt{y}}^{\sqrt{y}} \dfrac{1}{2} e^{-|x|} \, dx \\
&= \int_0^{\sqrt{y}} e^{-x} \, dx = 1 - e^{-\sqrt{y}}
\end{align*} \pause 
Thus, 
\begin{align*}
F_Y(y) = \begin{cases}
1 - e^{-\sqrt{y}} \quad &y \geq 0 \\
0 \quad &\text{otherwise}
\end{cases}
\end{align*}

\end{enumerate}
\end{frame}



\begin{frame}{Problem 3}
\begin{alertblock}{Problem 3}
Consider the PDF of the continuous random variable \pause 
\begin{align*}
f_X(x) = \begin{cases}
4x^3 \quad &0<x \leq 1 \\
0 \quad &\text{otherwise}
\end{cases}
\end{align*} \pause 
Find $P(X \leq \dfrac{2}{3} \mid X > \dfrac{1}{3}).$
\end{alertblock}
\end{frame}


\begin{frame}{Solution to Problem 3}
We have 
\begin{align*}
P(X \leq \dfrac{2}{3} \mid X > \dfrac{1}{3}) &= \dfrac{P(1/3 < X \leq 2/3)}{P(X > \dfrac{1}{3})} \\
&= \dfrac{\int_{1/3}^{2/3} 4x^3 \, dx}{\int_{1/3}^1 4x^3 \, dx} \\
&= 3/16
\end{align*}
\end{frame}



\begin{frame}{Problem 4}
\begin{alertblock}{Problem 4}
Consider the PDF of random variable $X$ \pause 
\begin{align*}
f_X(x) = \begin{cases}
x^2 (2x + \dfrac{3}{2}) \quad &0<x \leq 1 \\
0 \quad &\text{otherwise}
\end{cases}
\end{align*} \pause 
If $Y = \dfrac{2}{X} +3,$ find $\text{Var}(Y).$
\end{alertblock}
\end{frame}


\begin{frame}{Solution to Problem 4}
\begin{itemize}
\item We have 
\begin{align*}
\text{Var}(Y) &= \text{Var}(\dfrac{2}{X} + 3) = 4 \, \text{Var}(\dfrac{1}{X})
\end{align*} \pause 
\item We now find $\text{Var}(\dfrac{1}{X}) = E[\dfrac{1}{X^2}] - (E[X])^2$ \pause 
\item We have 
\begin{align*}
E[\dfrac{1}{X}] &= \int_0^1 x (2x + \dfrac{3}{2}) \, dx = \dfrac{17}{12} \\
E[\dfrac{1}{X^2}] &= \int_0^1 (2x + \dfrac{3}{2}) \, dx = \dfrac{5}{2}
\end{align*} \pause 
\item Hence, 
\begin{align*}
\text{Var}(\dfrac{1}{X}) = \dfrac{71}{144}
\end{align*} \pause 
\item $\text{Var}(Y) = 4 \, \text{Var}(\dfrac{1}{X}) = \dfrac{71}{36}$
\end{itemize}
\end{frame}


\begin{frame}{Problem 5}
\begin{alertblock}{Problem 5}
Let $X \sim \text{Uniform}(-\dfrac{\pi}{2}, \pi)$ and $Y = \sin(X).$ Find $f_Y(y).$
\end{alertblock} \pause 
%\vspace{4cm}
\begin{itemize}
\item Here $Y = g(X),$ where $g$ is a differentiable function \pause 
\item $g$ is not monotone, but it can be divided to a finite number of regions in which it is monotone \pause 
\item We have $R_X = [-\pi/2, \pi], \quad R_Y = [-1,1]$
\end{itemize}
\vspace{3cm}

\end{frame}


\begin{frame}{Answer to previous problem 5...}
%\hspace{5pt} {\color{red} \vrule width 1pt} \hspace{5pt}%
\hspace{8cm} \rule{.1mm}{1.0\textheight}
\end{frame}


\begin{frame}{Answer to previous problem 5...}

%\hspace{5pt} {\color{red} \vrule width 1pt} \hspace{5pt}%
\hspace{8cm} \rule{.1mm}{1.0\textheight}
\end{frame}






























